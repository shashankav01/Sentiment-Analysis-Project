\documentclass[12pt]{article}
\usepackage[utf8]{inputenc}
\usepackage{graphicx}
\usepackage{listings}
\usepackage{xcolor}
\usepackage{hyperref}
\usepackage{geometry}
\usepackage{float}

\geometry{a4paper, margin=1in}

\title{Mood Analyzer Project Report}
\author{Student Name}
\date{\today}

\begin{document}

\maketitle
\tableofcontents
\newpage

\section{Executive Summary}
This project implements an advanced mood analysis system that combines multiple natural language processing techniques to analyze and predict emotional content in text. The system utilizes machine learning, sentiment analysis, and emotion detection to provide comprehensive mood analysis.

\section{Introduction}
\subsection{Project Overview}
The Mood Analyzer is a sophisticated text analysis tool that combines various NLP techniques to understand and analyze emotional content in text. It integrates multiple approaches including:
\begin{itemize}
    \item Machine Learning-based mood prediction
    \item Sentiment Analysis using TextBlob and VADER
    \item Emotion keyword detection
    \item Emoji analysis
\end{itemize}

\section{Technical Architecture}
\subsection{Core Components}
\begin{itemize}
    \item TF-IDF Vectorizer for text feature extraction
    \item Random Forest Classifier for mood prediction
    \item NLTK and TextBlob for text processing
    \item VADER Sentiment Analyzer
    \item Custom emotion detection system
\end{itemize}

\subsection{Dependencies}
The project relies on several key Python libraries:
\begin{lstlisting}[language=Python]
import nltk
from textblob import TextBlob
from sklearn.feature_extraction.text import TfidfVectorizer
from sklearn.ensemble import RandomForestClassifier
from sklearn.preprocessing import LabelEncoder
from nltk.sentiment.vader import SentimentIntensityAnalyzer
\end{lstlisting}

\section{Implementation Details}
\subsection{Text Preprocessing}
The system implements robust text preprocessing:
\begin{itemize}
    \item Lowercase conversion
    \item Emoji handling
    \item Special character removal
    \item Tokenization
    \item Stopword removal
    \item Negation handling
\end{itemize}

\subsection{Machine Learning Model}
The Random Forest Classifier is configured with:
\begin{itemize}
    \item 300 estimators
    \item Maximum depth of 15
    \item Balanced class weights
    \item Cross-validation for performance evaluation
\end{itemize}

\subsection{Emotion Detection System}
Implements a comprehensive emotion keyword dictionary covering:
\begin{itemize}
    \item Joy
    \item Sadness
    \item Anger
    \item Fear
    \item Surprise
    \item Love
\end{itemize}

\section{Features Analysis}
\subsection{Sentiment Analysis}
The system provides multiple layers of sentiment analysis:
\begin{itemize}
    \item Basic sentiment scoring (-1.0 to 1.0)
    \item Subjectivity analysis
    \item VADER sentiment metrics
    \item Granular mood categories
\end{itemize}

\subsection{Mood Classification}
Implements a sophisticated mood mapping system:
\begin{lstlisting}
mood_mapping = {
    (-1.0, -0.7): "Very Negative",
    (-0.7, -0.4): "Negative",
    (-0.4, -0.1): "Slightly Negative",
    (-0.1, 0.1): "Neutral",
    (0.1, 0.4): "Slightly Positive",
    (0.4, 0.7): "Positive",
    (0.7, 1.0): "Very Positive"
}
\end{lstlisting}

\section{Performance Metrics}
\subsection{Model Evaluation}
The system includes built-in performance evaluation:
\begin{itemize}
    \item Cross-validation scoring
    \item Confidence metrics for predictions
    \item Probability distribution across mood categories
\end{itemize}

\section{Language Features Analysis}
The system provides detailed text analysis metrics:
\begin{itemize}
    \item Word count
    \item Sentence count
    \item Average word length
    \item Emotion intensity scoring
\end{itemize}

\section{Future Improvements}
\subsection{Potential Enhancements}
\begin{itemize}
    \item Integration of more advanced NLP models
    \item Expansion of emotion keyword dictionary
    \item Multi-language support
    \item Real-time analysis capabilities
\end{itemize}

\section{Conclusion}
The Mood Analyzer project demonstrates a sophisticated approach to text-based emotion analysis, combining multiple techniques and technologies to provide comprehensive mood analysis capabilities. The system's modular architecture and extensive feature set make it a valuable tool for understanding emotional content in text.

\end{document}